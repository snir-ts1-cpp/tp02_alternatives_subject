\documentclass[10pt]{article}
\usepackage{../_macrosLatex/macros}
\usepackage[a4paper,margin=2.5cm]{geometry}
\usepackage{fancyhdr} % Gestion header/footer

%----------------------------------------------------
% Paramétrage de la fiche
%----------------------------------------------------
\matiere{BTS SNIR TS1}
\sequence{Programmation en langage C++}
\seqlogo{\faCode}
\titrefiche{TP2 - Alternatives}
\version{v1.0}
\dateversion{21.09.22}
\type{TP}

%----------------------------------------------------
% Définition des pieds et têtes de page
%----------------------------------------------------
% Pour toutes les pages
\pagestyle{fancy}
\fancyhead[L]{\seqlogo\ \sequenceVal}
\fancyhead[R]{\titreficheVal}
\fancyfoot[L]{\matiereVal - Lycée Louis Rascol, Albi}
\fancyfoot[R]{\ccbyncsaeu}
\fancyfoot[C]{\thepage\ / \pageref{LastPage}}
% Pour la première page
\fancypagestyle{firstpage}{%
  \lhead{}
  \rhead{}
  \renewcommand{\headrulewidth}{0pt}
}

%----------------------------------------------------
% Début du document
%----------------------------------------------------
\begin{document}
\cartouche
\thispagestyle{firstpage}

\section*{Introduction}
Dans ce nouveau TP nous allons utiliser les structures alternatives vues en cours tel que : \mintinline{cpp}`if()`,
\mintinline{cpp}`if() ... else` et \mintinline{cpp}`switch() ... case`.

\begin{enumerate}
    \item Chaque exercice devra être \textbf{codé dans un projet CLion différent}.
    \item \textbf{Le code résidera dans le fichier main.cpp}
    \item Pour commencer copiez-coller le squelette de base d'un code C++ \textbf{dans le fichier main.cpp}.
    \item Remplissez correctement le cartouche pour chaque exercice.
\end{enumerate}

\begin{cppcode}
    /************************************************
    Nom du fichier : 
    Description : 
    Auteur :
    Date :
    BTS SNIR - TS1
    ************************************************/

    #include <iostream>
    using namespace std;

    int main() {
        // Mettez ici votre code C++
        return 0;
    }
\end{cppcode}


%----------------------------------------------------
% EXERCICES
%----------------------------------------------------
\section{Exercices}

\subsection{Pair ou impair}
Les entiers qui sont divisibles par 2 (division parfaite sans reste) sont appelés nombres pairs, les autres pour lesquels
la division par 2 donne un reste, sont appelés nombres impairs.
\smallskip
Afin de vérifier si un nombre est pair ou impair, il suffit de regarder le \textbf{reste de la division Euclidienne} (division quotient/reste), si le \textbf{reste vaut 0}, c'est que le nombre est \textbf{pair}, \textbf{sinon} c'est qu'il est \textbf{impair}. L'opérateur à utiliser est le modulo : \mintinline{cpp}`%`.

\begin{enumerate}
    \item En C++ déclarez la variable qui contiendra l'\textbf{entier} saisi par l'utilisateur.
    \item Demandez à l'utilisateur de saisir l'entier avec \mintinline{cpp}`cout`. 
    \item Occupez-vous de capturer la saisie clavier avec \mintinline{cpp}`cin` et de la stocker dans la variable créée. 
    \item Créez une structure \mintinline{cpp}`if() ... else`, dans la condition vous testerez si le nombre est pair ou impair.
    \item Si le nombre est pair afficher à l'écran \verb|nombre pair|, sinon \verb|nombre impair|.
\end{enumerate}


\subsection{Années bissextiles}
Une année est bissextile si elle est divisible par 4 mais pas par 100, ou si elle est divisible par 400. Vous créerez un code C++ qui demande à l'utilisateur de saisir une année et renvoi à l'écran si elle est bissextile ou non.

\smallskip
Par exemple 1200, 1600, 1968, 2000, 2004, 2012 étaient des années bissextiles.

\begin{itemize}
    \item En C++ déclarez la variable \mintinline{cpp}`annee` qui contiendra \textbf{l'année} saisie par l'utilisateur.
    \item Demandez à l'utilisateur de saisir l'année avec \mintinline{cpp}`cout`. 
    \item Occupez-vous de capturer la saisie clavier avec \mintinline{cpp}`cin` et de la stocker dans la variable créée.
    \item Créez une structure \mintinline{cpp}`if() ... else`, dans la condition vous testerez si l'année' est bissextile.
    \item Si elle est bissextile vous afficherez :\\
     \verb|[annee] est bissextile|\\
    sinon \\
    \verb|[annee] n'est pas bissextile| 
\end{itemize}


\subsection{Voyelle ou consonne}
Dans l'alphabet 6 lettres sont des voyelles : a, e, i, o, u, y, toutes les autres sont des consonnes.     
\smallskip
Vous devrez créer un programme qui demande à l'utilisateur de saisir une lettre et renvoyer à l'écran si celle-ci est une voyelle ou une consonne. \textbf{Attention, le test devra fonctionner si le caractère est une minuscule ou une majuscule}.

\begin{enumerate}
    \item En C++ déclarez la variable \mintinline{cpp}`car` qui contiendra \textbf{le caractère} saisi par l'utilisateur.
    \item Demandez à l'utilisateur de saisir le caractère avec \mintinline{cpp}`cout`. 
    \item Occupez-vous de capturer la saisie clavier avec \mintinline{cpp}`cin` et de la stocker dans la variable créée.
    \item Créez une structure \mintinline{cpp}`if() ... else`, dans la condition vous testerez si le caractère est une voyelle : Si il est égal à \mintinline{cpp}`'a'` ou \mintinline{cpp}`'e'`, ou \mintinline{cpp}`'i'`, ou \mintinline{cpp}`'o'`, ou \mintinline{cpp}`'u'`, ou \mintinline{cpp}`'y'`.
    \item Si le caractère est une voyelle, vous afficherez :\\
     \verb|[car] est une voyelle|\\
    sinon \\
    \verb|[car] est une consonne|
\end{enumerate}


\subsection{Plus grand parmi 3}
Vous demanderez à l'utilisateur de saisir en une fois 3 nombres qui peuvent être des décimaux. Votre programme
trouvera le nombre le plus grand parmi les 3 et l'affichera à l'écran.

\begin{enumerate}
    \item En C++ déclarez les variables qui contiendront \textbf{les nombres décimaux} saisis par l'utilisateur : \mintinline{cpp}`a`, \mintinline{cpp}`b` et \mintinline{cpp}`c`  
    \item Demandez à l'utilisateur de saisir les 3 nombres décimaux avec \mintinline{cpp}`cout`. 
    \item Occupez-vous de capturer la saisie clavier avec \mintinline{cpp}`cin` et de la stocker dans les variables créées.
    \item Créez \textbf{3 structures} \mintinline{cpp}`if()` : la première testera si \mintinline{cpp}`a` est le plus grand, la seconde si \mintinline{cpp}`b` est le plus grand et la dernière si \mintinline{cpp}`c` est le plus grand.
    \item Pour chaque \mintinline{cpp}`if()` vous afficherez à l'écran le nombre trouvé s'il est reconnu comme étant le plus grand :\\
    \verb|Le nombre [a, b ou c] est le plus grand|
\end{enumerate}

\subsection{Résultats d'une équation du premier degré}
Une équation du premier degré : $ax^2+bx+c$ se résout en calculant le discriminant $\Delta$ : 
$$\Delta=b^2-4ac$$
Suivant le signe de $\Delta$, il y a 3 possibilités :
\begin{itemize}
    \item $\Delta<0$ : Pas de solution dans les $\mathbb{R}$
    \item $\Delta=0$ : Une solution double : $$x_1=x_2=\frac{-b}{2a}$$
    \item $\Delta>0$ : Deux solutions : $$x_1=\frac{-b-\sqrt{\Delta}}{2a} \qquad x_2=\frac{-b+\sqrt{\Delta}}{2a}$$
\end{itemize}

Vous ferez un code C++ qui demande à l'utilisateur de saisir les coefficients de l'équation : \mintinline{cpp}`a, b, c` et renverra à l'écran les solutions de l'équation.

\begin{itemize}
    \item En C++ déclarez les variables qui contiendront \textbf{les coefficients entiers} saisis par l'utilisateur : \mintinline{cpp}`a`, \mintinline{cpp}`b` \mintinline{cpp}`c` et \mintinline{cpp}`delta` 
    \item Demandez à l'utilisateur de saisir les 3 nombres décimaux avec \mintinline{cpp}`cout`. 
    \item Occupez-vous de capturer la saisie clavier avec \mintinline{cpp}`cin` et de la stocker dans les variables créées.
    \item Calculez le discriminant, stockez le résultat dans \mintinline{cpp}`delta`
    \item Créez \textbf{3 structures} \mintinline{cpp}`if()` : La première testera si \mintinline{cpp}`delta` est inférieur à 0, la suivante s'il est supérieur et la dernière s'il est nul.
    \item Pour chaque \mintinline{cpp}`if()` vous ferez le calcul de la solution et afficherez le résultat.
\end{itemize}


\subsection{Calculatrice}
Votre programme demandera à l'utilisateur de choisir un opérande parmi les 4 calculs de bases : \mintinline{cpp}`+, -, x, /`, 
puis de saisir 2 nombres (pouvant être des décimaux). Immédiatement après la saisie il affichera le résultat de l'opération
choisie.

Exemple d'affichage sur la console :
\begin{textcode}
    Choisissez votre opérateur : +, -, *, /: -    
    Entrez 2 opérandes : 3.4 8.4
    3.4 - 8.4 = -5
\end{textcode}

\begin{itemize}
    \item En C++ déclarez les variables qui contiendront \textbf{les deux nombres décimaux} saisis par l'utilisateur : \mintinline{cpp}`a` et \mintinline{cpp}`b` ainsi que le \textbf{caractère} correspondant à l'opération choisie \mintinline{cpp}`operateur` et le décimal stockant le résultat : \mintinline{cpp}`rslt` 
    \item Demandez à l'utilisateur de saisir l'opérateur avec \mintinline{cpp}`cout`. 
    \item Occupez-vous de capturer la saisie clavier avec \mintinline{cpp}`cin` et de la stocker dans la variable \mintinline{cpp}`operateur`.
    \item Demandez à l'utilisateur de saisir les 2 opérandes avec \mintinline{cpp}`cout`. 
    \item Occupez-vous de capturer la saisie clavier avec \mintinline{cpp}`cin` et de la stocker dans les variables \mintinline{cpp}`a` et \mintinline{cpp}`b`.
    \item Vous mettrez en place une structure \mintinline{cpp}`switch()...case` avec un \mintinline{cpp}`case` par opération.
    \item Programmez les opérations dans chaque case et affectez le résultat à la variable \mintinline{cpp}`rslt`. 
    \item Affichez le calcul et son résultat correspondant.
\end{itemize}    

\end{document}